\documentclass[12pt]{article}
%\usepackage[a4paper, margin=2cm]{geometry}
\usepackage[utf8]{inputenc}
\usepackage[T1]{fontenc}
\usepackage{url}
\usepackage{breakurl}
\usepackage[breaklinks]{hyperref}
\def\UrlBreaks{\do\/\do-}
\usepackage[backend=biber]{biblatex}
\usepackage[ngerman]{babel}
\usepackage{svg}
\usepackage{float}
\usepackage{wrapfig}
\usepackage[german,onelanguage]{algorithm2e}
\usepackage{mathptmx}
\usepackage{fancyhdr}

\usepackage{ifpdf}
\usepackage{graphicx}
\ifpdf
  \DeclareGraphicsRule{*}{mps}{*}{}
\fi


\SetKwProg{Fn}{Funktion}{ ist}{Ende}

\setcounter{biburllcpenalty}{7000}
\setcounter{biburlucpenalty}{8000}

\pagestyle{fancy}

 
\addbibresource{Extras/sources.bib}

\title{Wege im Aufbauspiel \linebreak Ant Colony Optimization}

\author{Adrian Kumbrink}

\pagenumbering{roman}

\begin{document}

\begin{titlepage}


\maketitle
\thispagestyle{empty}

\vspace*{\fill}
\begin{center}
Informatik GK Buyken - Conrad von Soest Gymnasium
\end{center}

\end{titlepage}



\thispagestyle{empty}
\tableofcontents
\pagebreak


\section{Einleitung} \label{Einleitung}
\pagenumbering{arabic}
Bei der Entwicklung von Algorithmen stellt sich oft die Frage nach der besten Lösung. Es gibt tausende von Möglichkeiten ein gegebenes Problem anzugehen. Eine der besten Inspirationsquellen ist die Natur selbst. Mit ihren erprobten Methoden liefert sie oft Vorbilder. Eins dieser Vorbilder sind Ameisenkolonien. Sie geben Ideen für das Kommunizieren, für das Konstruieren und eben auch für das Wegfinden.\\\\
Ein typisches Problem ist die Routenplanung, wo in einem Netzwerk (Graph) der optimalste Weg gefunden werden soll. Ant Colony Optimization (auch ACO) nimmt ein solches Problem als Graphen und findet mögliche Lösungen zu eben diesem Problem. Hierbei hat die Futtersuche der Ameisen als Inspiration gedient. In der Natur zeigen sie durch ihre Schwarmintelligenz, dass viele einzelne Ameisen die Weite ihres Lebensraumes auf wenige effizient gewählte Straßen vereinfachen können. So hinterlassen sie, auf der Suche nach Nahrung, Pheromone, denen dann wiederum anderen Ameisen folgen und daher Straßen bilden. Dadurch können relativ simple Abfolgen komplexe Ziele erreichen.\\\\
Aber warum jetzt Wegfindung in Aufbausimulationen? In Aufbausimulationen müssen oft Wege zwischen verschiedenen Punkten gefunden werden. Daher bietet sich hier ACO an. Besonders oft werden Waren zwischen unterschiedlichen Produktionsstätten transportiert und dann ist der Träger nichts anderes als die Ameise, der Startpunkt der Ameisenbau und das Ziel die Nahrung. Aufgrund der iterativen Natur der ACO bleiben die Routen nicht statisch, aber erprobte Wege bleiben erhalten.
\section{Der Algorithmus}
\subsection{Die Geschichte}\label{Geschichte}
Der Ant Colony Optimization Algorithmus wurde erstmalig von Marco Dorigo in seiner Doktorarbeit 1992 vorgestellt. Diese Variante ist als das Ant System (AS) bekannt. Seitdem gab es einige Weiterentwicklungen des ursprünglichen Algorithmus, wie zum Beispiel das Ant Colony System oder das Elitist Ant System.
Ursprünglich wurde das Ant System anhand des travelling salesman problem (TSP) (s. \ref{TSP}) vorgestellt.
\subsection{Algorithmus}
Ein Algorithmus ist ein"\textit{ Rechenvorgang nach einem bestimmten [sich wiederholenden] Schema}"\cite{duden_algo}. Daher werden in den folgenden Abschnitten die benötigten Abläufe für ACO definiert.
\subsection{Graphen}
\begin{wrapfigure}{r}{0.55\textwidth}
\centering
\includeinkscape[width=0.5\textwidth ,height=0.5\textwidth]{Extras/Abb1.pdf_tex}
\caption{Einfacher Graph}
\label{fig:1}
\end{wrapfigure}
Als Graphen versteht man eine Anzahl an Knoten(V\textsubscript{x}) und Kanten(E\textsubscript{0}). Eine Kante zwischen zwei Knoten wird als (V\textsubscript{a},V\textsubscript{b}) bezeichnet, wobei V\textsubscript{a} und V\textsubscript{b} die beiden verbundenen Knoten sind (Abbildung \ref{fig:1}).Das Bewegen zwischen V\textsubscript{a} nach V\textsubscript{b} über (V\textsubscript{a}, V\textsubscript{b}) wird durch die Kosten bewertet. Diese sind im einfachsten Fall die Länge der Kante. Werden den Kanten ein bzw. zwei Richtungen zugewiesen, so spricht man von einem gerichteten Graphen. Weiterhin kann den Kanten oder den Knoten ein Gewicht, also ein Kostenmultiplikator für die jeweilige Kante bzw. Knoten zugewiesen werden. Bei ersterem spricht man von einem kantengewichteten Graphen, bei letzterem von einem knotengewichteten Graphen.\\\\
Zwei Knoten gelten als benachbart, wenn es (V\textsubscript{a},V\textsubscript{b}) oder (V\textsubscript{b},V\textsubscript{a}) gibt.
\subsection{Die Kernidee}\label{Kernidee}
"'\textit{In ACO, a number of artificial ants build solutions to an optimization problem and exchange information on their quality via a communication scheme that is reminiscent of the one adopted by real ants.}"\cite{ dorigo2007ant}\footnote{DE: Bei dem Ant Colony Optimization Algorithmus, entwickeln künstliche Ameisen eine Lösung zu einem Optimierungsproblem und kommunizieren die Qualität ihrer Lösung via eines Verfahrens, welches von Ameisen adaptiert wurde.  }. 
Die Ant Colony Optimization besteht grob aus zwei Bestandteilen, der Wahrscheinlichkeit für eine gegebene Kante und der Pheromonenupdateregel. Die hier vorgestellten Formeln entsprechen der des Ant Systems (AS) (s. \ref{Geschichte}).
\subsubsection{Wahrscheinlichkeit für eine Kante}\label{P/p}
Die Wahrscheinlichkeit für eine Kante gibt an wie hoch die Wahrscheinlichkeit (p) ist, dass eine gegebene Ameise ($k$) die Kante (V\textsubscript{x},V\textsubscript{y}) nimmt, wobei V\textsubscript{x} die momentane Position ist. Sie setzt sich aus der Menge an Pheromonen, der Attraktivität,  auf der Kante ($\tau$) und der Effektivität der Kante ($\eta$) zusammen. Der Einfluss der beiden Faktoren kann durch den Nutzer in $\alpha$ bzw. $\beta$ beeinflusst werden. \[ P ^{ k }_{ xy }=\left(  \tau ^{ \alpha }_{ xy } \right)*\left(  \eta ^{ \beta }_{ xy } \right)\]
Der Wert wird dann in eine relative Wahrscheinlichkeit umgerechnet, indem $P^k_{xy}$ durch die Summen der absoluten Wahrscheinlichkeiten aller möglichen Kanten (V\textsubscript{x},V\textsubscript{z}) mit V\textsubscript{z} als $V_z \in Nachbarn_{x}$ geteilt wird.Daher ist: \[p^{k}_{xy}=\frac{P^k_{xy}}{\sum_{z}^{Nachbarn_{x}}{P^{k }_{xz}}} \]
\subsubsection{Die Pheromonenupdateregel}
Der zweite Teil ist die Pheromonenupdateregel. Sie ist für das Aktualisieren aller Kanten zuständig. Daher hat diese Regel einen großen Einfluss auf das Verhalten der Ameisen. Die einfachste Regel besagt: \[\tau_{xy}= (1-\rho)*\tau_{xy}+\sum_{k=1}^{m}{ \Delta\tau^k_{xy}} \]
Wobei $\rho$ angibt wie viel von dem Duftstoff verdampft, $m$ die Anzahl an Ameisen ist und $\Delta\tau^k_{xy}$ die Menge an Pheromonen ist, die die Ameise $k$ auf der Kante (V\textsubscript{x},V\textsubscript{y}) hinterlassen hat. Dementsprechend ergibt sich $\Delta\tau^k_{xy}$ wie folgt, wenn die Ameise $k$ über die Kante gegangen ist:\[\Delta\tau^k_{xy}=Q/L_k\] L\textsubscript{k} sind die Kosten, die die Ameise k aufwenden musste, um die Kante (V\textsubscript{x},V\textsubscript{y}) zu passieren. Q ist ein weiterer Faktor, mit dessen Hilfe der Nutzer den Algorithmus beeinflussen kann. Wenn die Ameise nicht über die Kante gegangen ist, dann ist: \[\Delta\tau^k_{xy}=0\]
\pagebreak
\subsection[TSP]{Das travelling salesman problem}\label{TSP}
Das travelling salesman problem (in deutsch: Problem des reisenden Händlers) ist ein relativ bekanntes Problem 
in der Computertechnik. Es stellt die Frage:
"'\textit{Given a list of cities and the distances between each pair of cities, what is the shortest possible route that visits each city exactly once and returns to the origin city?}"\cite{wiki_TSP}\footnote{DE: Wenn eine Liste an Städten und deren Entfernungen zueinander bekannt ist, was ist dann der kürzeste Weg, um alle Städte genau einmal zu besuchen und zum Anfang zurück zu kehren?}. Der Ursprung dieser Fragestellung ist unbekannt.\\\\
Soll dieses Problem mit Hilfe von AS gelöst werden, so entspricht jede Stadt einem Knoten und alle Knoten sind untereinander verbunden. Danach können die oben beschriebenen Regeln angewandt werden (s. \ref{Kernidee})
Als Beispiel soll der Weg in einem Netzwerk mit sechs Städten gefunden werden. Der Graph sähe dann wie in Abbildung \ref{fig:2} aus. An jeder Stadt wird eine Ameise gestartet. Der kürzeste Weg \cite{aco-sim} sieht dann wie folgt aus (s. Abbildung \ref{fig:2} rote Linien).
\begin{wrapfigure}[13]{l}{0.55\textwidth}
\centering
\includeinkscape[width=0.5\textwidth ,height=0.5\textwidth]{Extras/Abb2.pdf_tex}
\caption{Netzwerk an Städten}
\label{fig:2}
\end{wrapfigure}
\subsection{Die Anwendung}
Nachdem nun der theoretische Teil betrachtet wurde, sind natürlich auch die tatsächlichen Anwendungen für ACO interessant. Einer der offensichtlichen Bereiche ist, der der Wegfindung. So wird ACO unter anderem in der Netzwerktechnik genutzt um Wege für die Datenpakete zu finden. Aber auch in den Naturwissenschaften findet ACO Anwendung, zum Beispiel in der Biologie zur Proteinfaltung oder in der Nanotechnik. 
\section{Anwendungen im Aufbauspiel}
Nachdem nun der Algorithmus betrachtet wurde, folgt jetzt die Übertragung auf den Bereich der Aufbauspiele.

\subsection{Das Problem}
Ein Aufbauspiel soll aus zwei Teilen bestehen. Der erste sind die Gebäude die Waren verbrauchen und produzieren. Der zweite Teil sind die Waren selbst. Die Gebäude bestehen aus einer Liste an benötigten und produzierten Waren und einer Produktionsgeschwindigkeit. Diese gibt, an nach welcher Zeit die produzierten Waren in das Inventar hinzugefügt werden. Bedingung dafür ist, dass die benötigten Waren vorhanden sind und im Inventar Platz für die produzierten ist. Die Größe des Inventars wird durch die Inventarkapazität angegeben.
Die Waren bestehen nur aus ihrem Volumen, das beschreibt wie viel Platz sie in dem Inventar einnehmen.\\\\
Als Ausgangslage dient das Bauen von Produktionsketten. Da es unterschiedlichste Ketten mit variierender Komplexität gibt, soll hier beispielhaft nur die der Bretterproduktion betrachtet werden.\\\\
Daher ergeben sich zwei notwendige Gebäude. Ersteres ist die Holzfällerhütte, die in einem festen Zeitintervall von 2t\footnote{t: beliebige Zeit} einen Holzstamm\footnote{Platz eingenommen: 2 VE} produziert, ohne dabei benötigte Ressourcen\footnote{Das Fällen von Bäumen und deren Existenz werden zur Vereinfachung weggelassen.} zu haben. Die Lagerkapazitäten des Holzfällers sind 1L\footnote{L: Volumen des Lagers}. Das zweite Gebäude ist das Sägewerk, das alle 1t\footnotemark[3] aus einem Holzstamm ein Brett\footnote{Platz eingenommen: 1 VE} produziert. Das Sägewerk hat eine Lagerkapazität von 2L\footnotemark[6]. 
Als drittes Gebäude wird noch das Lagerhaus hinzugefügt. Es dient nur dazu die produzierten Waren zu sammeln und die benötigten auf die Verbraucher zu verteilen. Weiterhin können die Lagerhäuser auch als Zwischenlager verwendet werden, um überschüssige Waren zu lagern. Außerdem dienen sie den Ameisen als Ziel und Ausgangspunkt ihrer Reise.\\\\
Die unterschiedlichen Gebäude werden durch Straßen miteinander verbunden.
\subsection{Die ACO Regeln}\label{Regeln}
Damit der Standard Ant System Algorithmus für Aufbauspiele funktioniert, müssen gewisse Regeln erweitert werden. \\\\
Die Regel für die Effektivität $\tau$ besteht nicht nur aus der Distanz, damit die Ameisen nicht die Waren über die ganze Karte tragen, sondern auch aus der Bilanz des Zieles. Die Bilanz beschreibt, wieviel Ressourcen die benachbarten Gebäude in einer Zeiteinheit t\footnotemark[3] verbrauchen (negativ) und produzieren (positiv). Entsteht ein Überschuss an einer Ware so ist deren Wert positiv. Wird mehr als vorhanden benötigt, so ist der Wert negativ. Diese Regel soll verhindern, dass die Ameisen ihre Ware zu einem Lagerhaus bringen, das einen Überschuss an dieser hat. Der Einfluss von der Bilanz kann von dem Nutzer durch $\theta$ beeinflusst werden.

\subsection{Die Implementation}\label{pseudo}
Nachdem die Grundlage für das Spiel gesetzt wurde, sollen nun die einzelnen Formeln und Abläufe in Pseudocode übersetzt und erläutert werden. \\\\
Ein zentraler Teil des Algorithmus ist die Berechnung der absoluten Wahrscheinlichkeit $P$. \\\\
\begin{algorithm}[H]
 \caption{Wahrscheinlichkeit P}
 \DontPrintSemicolon

 \Fn{berechneP(kante : Kante) : Dezimalzahl}{
 P = (kante.Attraktivität * $\alpha$ ) * (effektivität(kante) * $\beta$) : Dezimalzahl\;
 gib P zurück\;
 }
\end{algorithm}Diese Funktion ist eine programmatische Umsetzung, der unter \ref{P/p} beschriebenen Formel für P.
Die Funktion effektivität(kante), den Regeln aus \ref{Regeln}  folgend, verläuft wie folgt:\\\\
\begin{algorithm}[H]
\caption{Effektivität}
\DontPrintSemicolon
\Fn{effektivität(kante : Kante) : Dezimalzahl}{
ziel = null : Knoten\;
\eIf{kante.A gleich momentanePosition}{
	ziel = kante.B\;
}{
ziel = kante.A\;
}
d = $\sqrt[2]{(kante.A.x - kante.B.x)^2 + (kante.A.y - kante.B.y)^2}$ : Dezimalzahl\;
\uIf{ziel.gebäude ist Variante von Lagerhaus und ziel.gebäude.balanceEnthält(transportierteRessource) }{
d = d + $\theta$ * ziel.gebäude.balance(transportierteRessource)\;
}
gib $\frac{1}{d}$ zurück\;
}
\end{algorithm}Im ersten Teil der Funktion wird entschieden, welcher der beiden Knoten der Kante genommen und betrachtet werden soll. Da nicht zu der eigenen Position gegangen werden soll, müssen daher die Werte des anderen Knotens als Zielwerte verwendet werden. Als nächster Schritt ist dann die Distanz zwischen den beiden Knoten zu berechnen. Sollte das Gebäude an dem Zielknoten ein Lagerhaus sein, so wird der Verbrauch bzw. die Produktion der transportierten Ressource auf die Distanz ab- bzw. aufgeschlagen. 
Der Rückgabewert entspricht 1/d, damit größere Strecken unattraktiver sind.\\\\
Im nächsten Schritt muss die relative Wahrscheinlichkeit p berechnet werden. Dazu wird die absolute Wahrscheinlichkeit P durch die Summe aller Wahrscheinlichkeiten geteilt.\\\\
\begin{algorithm}[H]
\caption{Summe der Wahrscheinlichkeiten}
\DontPrintSemicolon
\Fn{summeWahrscheinlichkeiten() : Dezimalzahl}{
ergebnis = 0 : Dezimalzahl\;
\ForEach{momentan : Kante in momentanePosition.Kanten}{
ergebnis = ergebnis + berechneP(momentan)\;
}
gib ergebnis zurück\;
}
\end{algorithm}Dieser Algorithmus addiert die Ergebnisse der Funktion P für jede Kante, die von der momentanen Position ausgehen, iterativ.\\\\
Daher ergibt sich die relative Wahrscheinlichkeit p wie folgt:\\\\
\begin{algorithm}[H]
\caption{Wahrscheinlichkeit p}
\DontPrintSemicolon
\Fn{berechnep(kante : Kante) : Dezimalzahl}{
gib $\frac{berechneP(kante)}{summeWahrscheinlichkeiten()}$ zurück\;
}
\end{algorithm}Damit nun eine Kante selektiert werden kann, muss zuerst jeder, vom momentanen Knoten ausgehenden Kante, eine Wahrscheinlichkeit zugewiesen und dann eine Kante ausgewählt werden. Die Varianz des Algorithmus ergibt sich in dieser Auswahl. Hierzu wird zufällig ein Schwellwert zwischen 0 und 1 (0\% und 100\%) bestimmt. Dann werden alle errechneten Wahrscheinlichkeiten solange summiert bis dieser Wert überschritten ist. Diese Kante ist dann die Kante, welche die Ameise in der Runde gehen wird.
\begin{figure}[H]
\centering
\includeinkscape[width=0.4\textwidth ,height=0.4\textwidth]{Extras/Abb3.pdf_tex}
\caption{Aufbau der Liste an Wahrscheinlichkeiten}
\label{fig:3}
\end{figure}
Der Pseudocode sieht daher so aus:\\\\
\begin{algorithm}[H]
\caption{Selektion einer Kante}
\DontPrintSemicolon
\Fn{selektiereKante() : Kante}{
wahrscheinlichkeiten = Array mit Länge von momentanePosition.Kanten : Dezimalzahlarray\;
index = 0 : Zahl\;
\ForEach{momentan : Kante in momentanePosition.Kanten}{
wahrscheinlichkeiten[index] = berechnep(momentan)\;
index = index + 1\;
}

schranke = Zufallszahl zwischen 0 und 1 : Dezimalzahl\;
summe = 0 : Dezimalzahl\;
index = 0\;
\ForEach{momentan : Dezimalzahl in wahrscheinlichkeit}{
summe = summe + momentan\;
\If{summe ist größer als schranke}{
	gib momentanePosition.Kanten[index] zurück\;
}
index = index + 1\;
}
gib null zurück\;
}
\end{algorithm}Nachdem die Kanten für alle Ameisen bestimmt und diese auch gegangen worden sind, müssen nun die Pheromonen auf den Kanten neu verteilt werden. Dazu merkt sich das Programm, über welche Kante sich in der jetzigen Runde jeweils eine Ameise bewegt hat. Im ersten Teil wird dann die allgemeine Verdunstung berechnet.\\\\
\begin{algorithm}[H]
\Fn{verdunste() : nichts}{
\ForEach{momentanerKnoten : Knoten in alleKnoten}{
\ForEach{momentaneKante : Kante in momentanerKonten.Kanten}{
\If{momentaneKante.A gleich momentanerKnoten}{
momentaneKante.Attraktivität = momentatneKante.Attraktivität * ($1 - \rho$)\;
}
}
}
}
\caption{Verdunstung}
\end{algorithm}Als zweiter Schritt werden dann alle begangenen Kanten, entsprechend den Regeln, attraktiver gemacht.\\\\
\begin{algorithm}[H]
\DontPrintSemicolon
\Fn{belohne() : nichts}{
\ForEach{momentane : Kante in geganeneKanten}{
	d = 1 : Dezimalzahl\;
	d = $\sqrt[2]{(momentane.A.x - momentane.B.x)^2 + (momentane.A.y - momentane.B.y)^2}$ \;
	momentane.Attraktivität = momentane.Attraktivität * Q / d\;
}
}
\end{algorithm}
Hier werden der Einfachheit halber die Kosten mit der Länge gleichgesetzt.\\\\
Oben nicht gezeigt, aber dennoch von Bedeutung ist die Bewegung der Ameisen. Auch die Entscheidung welche Ressource transportiert werden soll, die Menge derer und der Abgabeort werden hier nicht weiter betrachtet.
\section{Bewertung}
\subsection{Nachteile}\label{Nachteile}
Natürlich hat die Anwendung des ACO Algorithmus auch ihre Nachteile. So ist er nicht deterministisch. Dies bedeutet, dass bei zwei exakt gleichen Graphen, unterschiedliche Wege entstehen können.
Jenes hat zur Folge, dass es eine optimale und reproduzierbare Lösung nicht geben kann. Für ein optimales Ergebnis müssen die Faktoren ($\alpha,\beta etc.$) experimentell ermittelt werden.\\\\
Da der Algorithmus die einzelnen Warentransporte simuliert, eignet er sich nicht für Simulationen mit großem Transportaufkommen.
\subsection{Vorteile}
Die Zufallskomponente des Algorithmus fördert die Spieldynamik. Da es eine absolut beste Lösung nicht gibt, lädt der Algorithmus den Spieler zu eigenem Experimentieren ein.
Weiterhin passt sich ACO neuen Gegebenheiten automatisch an. Aufgrund der immer fortwährenden Optimierung, erfolgt ein solches Anpassen nur langsam. Dies bildet reale Prozesse eher ab.
\subsection{Erweiterungen}
Der in \ref{pseudo} beschriebene Pseudocode bietet noch Möglichkeiten der Verbesserung.
Eine dieser wäre, die Berechnung der Wahrscheinlichkeit zu parallelisieren, da die Berechnungen der einzelnen Wahrscheinlichkeiten nicht voneinander abhängig sind. Auch eine Analyse des Algorithmus, besonders im Hinblick auf sprachliche Möglichkeiten, könnte sinnvoll sein.\\\\
Auch der Algorithmus selbst bietet Möglichkeiten für Erweiterung. So könnte die Berechnung der Pheromone so erweitert werden, dass viel genutzte Kanten gemieden werden. Diese Erweiterung hätte eine dynamischere Verteilung der Ameisen zur Folge. Auch eine Betrachtung der Umgebung des Zieles kann bessere Resultate liefern.
\section{Fazit}
Der ACO Algorithmus ist durchaus ein valide Option für Aufbauspiele. Allerdings gibt es einige Voraussetzungen. Soll das Spiel komplett vorhersehbar sein, so funktioniert der Algorithmus nicht. Ebenso bereitet er Probleme bei Spielen mit massiver Größe an Simulationen (wie z.B. Anno 1800).
Wird aber ein Spiel wie die Siedler angestrebt, in denen vielfältiger Warentransport erwünscht ist, ist ACO durchaus eine Überlegung wert.


\section{UMLs}
\begin{center}
\includegraphics[scale=0.9]{Extras/uml.0}\footnote{Die hier gezeigten Diagramme enthalten nur in dieser Arbeit explizit erwähnte Funktionen und Variablen.}
\end{center}
\nocite{*}
\printbibliography[heading=bibnumbered]
Quellcode, Orginaldateien können in \href{https://github.com/Kauruck/Facharbeit_AOC}{dem Github repository} eingesehen werden.

\end{document}
